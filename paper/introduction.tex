\section{Introduction} \subsection{Background and Motivation}
Human motion capture and analysis is a cornerstone of sports science and biomechanics, enabling detailed assessment of athletic performance and injury mechanisms. Traditionally, capturing human movement with high fidelity has required specialized hardware and controlled settings. For example, marker-based optical motion capture systems (such as Vicon) use multiple high-speed cameras and reflective markers attached to the body to triangulate 3D joint positions with high accuracy. While such systems are considered the gold standard for biomechanical analysis \cite{OpticalMocapLimit}, they are prohibitively expensive and confined to indoor laboratory environments. This makes them impractical for outdoor sports scenarios like football fields or running tracks, where large capture volumes and unconstrained settings are required. Similarly, depth-sensing cameras (e.g., the Microsoft Kinect) provide a markerless alternative but are limited to short-range indoor use and perform poorly in sunlight or expansive outdoor areas \cite{KinectLimit}. These constraints highlight the limitations of current indoor motion tracking technologies: high cost, complex setup, and lack of portability to real-world outdoor sports. Wearable sensor systems offer another approach to motion tracking. Inertial measurement unit (IMU) suits and similar wearable devices can record body kinematics without external cameras, and have been employed in outdoor training sessions. However, they too have significant drawbacks. High-end IMU-based motion capture suits require dozens of sensors strapped to the athlete’s body, calibration procedures, and data fusion algorithms to estimate joint angles. The financial cost of outfitting multiple athletes with such suits is substantial, and the process of wearing and calibrating sensors can be cumbersome \cite{Rekant2021}. Moreover, wearable systems, while freer from environmental restrictions, may still interfere with natural movement to some degree and typically only capture the instrumented individuals. In team sports or group exercises, equipping every participant with a full sensor suite is often impractical. As a result, current solutions are not easily scalable to multi-person scenarios in outdoor settings – a critical gap if one wishes to analyze an entire soccer team’s formation or multiple runners on a track simultaneously. These challenges motivate the exploration of camera-only, markerless motion capture solutions that can operate in outdoor environments. The vision of accessible, video-based human pose estimation is to use ordinary RGB cameras (e.g., a smartphone or DSLR camera) to capture athletes in action and automatically infer their body joint coordinates. Such a solution would eliminate the need for wearable gear or specialized labs, dramatically lowering the barrier to obtaining biomechanical data. Recent advances in computer vision and deep learning have indeed made markerless pose estimation feasible. Convolutional neural networks trained on large datasets of human images can now detect keypoints (e.g., elbows, knees, shoulders) on persons in 2D images with impressive accuracy \cite{Cao2017}. In particular, multi-person pose estimation algorithms like OpenPose have demonstrated the ability to identify poses of multiple people in an image simultaneously by learning to associate detected body parts with the correct individuals \cite{Cao2017OpenPose}. This progress opens the door to applying pose estimation in sports contexts; early studies have shown promise in using pose data for analyzing exercises and athletic motions \cite{Badiola2021}. However, a number of gaps remain before such vision-based systems can fully replace traditional motion capture for outdoor sports analysis. First, many state-of-the-art pose estimation models are computationally heavy, relying on deep CNN backbones (e.g., ResNet or HRNet) to achieve high accuracy. They often require powerful GPUs to run in real-time, which is not feasible for on-field use with portable devices. For instance, OpenPose can operate in real-time on a high-end desktop GPU, but deploying it on a mobile camera or edge device would result in low frame rates and latency. Recent research has started to focus on lightweight pose estimation networks that sacrifice some accuracy for speed, enabling real-time performance on consumer hardware. An example is BlazePose, a lightweight model optimized for smartphone processors, which demonstrates that real-time body tracking on mobile devices is achievable for a single person \cite{Bazarevsky2020}. Extending such efficiency to multi-person scenarios is an ongoing challenge: processing multiple athletes in one view increases the computational load, especially for top-down approaches that run a separate pose network per detected person. Bottom-up approaches (which detect all keypoints then group them) offer more scalable inference \cite{OpenPose}, but still need to be streamlined for resource-constrained environments. Thus, there is a clear need for research on efficient multi-person pose estimation algorithms that can maintain accuracy while operating in real-time with minimal hardware. Second, outdoor sports environments introduce complexities that are less pronounced in indoor or lab settings. Background clutter, varying lighting conditions (bright sun, shadows, glare), and weather conditions can all degrade camera-based pose estimation performance. Athletes may occupy a wide area in the frame, appear small at a distance, or occlude each other during interactions (e.g., tackles or group formations), making keypoint detection harder. Most existing pose estimation models are trained on general-purpose datasets (such as COCO or MPII) that contain everyday images and may not specifically represent athletic movements or field sports scenarios \cite{Suo2024}. Consequently, their performance can suffer when applied to sports footage, especially for uncommon poses or extreme joint angles seen in high-performance athletics. Data scarcity is an issue here: specialized datasets for outdoor sports pose estimation are only beginning to emerge (e.g., an annotated football pose dataset \cite{Nibali2021}), and adapting models to these contexts is non-trivial. This thesis is motivated by these challenges, aiming to bridge the gap between controlled-environment motion capture and in-the-wild sports motion analysis. We seek to make outdoor human motion analysis more accessible by leveraging camera-only systems, powered by tailored deep learning models that are accurate, efficient, and robust to real-world conditions. Ultimately, such technology could democratize sports biomechanics, allowing coaches and athletes at all levels to obtain lab-quality motion insights on the field, and enabling continuous monitoring for performance enhancement and injury prevention. \subsection{Research Objectives}
In light of the above motivation, the objectives of this research are defined as follows:
\begin{enumerate}
\item \textbf{Develop a lightweight multi-person pose estimation network for outdoor scenarios.} The first goal is to design and train a deep learning model capable of real-time human pose estimation from video in unconstrained outdoor environments. This involves creating an architecture that is computationally efficient (suitable for running on standard laptops or mobile devices) while still accurately detecting multiple athletes and their key joint positions in each frame.
\item \textbf{Analyze sports performance using markerless motion capture data.} By deploying the pose estimation system on sports footage, we aim to extract meaningful kinematic features and quantitative metrics of athletic performance. This includes calculating joint angles, limb velocities, and other biomechanical parameters from the pose sequences. The objective is to demonstrate that our camera-based approach can provide coaches and athletes with feedback on technique and performance comparable to traditional motion analysis, but with far greater convenience.
\item \textbf{Predict injury risk from captured motion patterns.} A key objective is to investigate how the pose-derived metrics can be used to assess injury risk factors in athletes. By identifying aberrant movement patterns or joint loading conditions (for example, excessive knee valgus during landing, which is associated with ACL injury risk), the system will attempt to flag potential injury risks before an injury occurs. This may involve developing a risk prediction model or heuristic that maps certain kinematic criteria to elevated injury risk, and validating it against known risk assessments or injury case studies. Ultimately, the goal is to show that integrating pose estimation with injury risk prediction can form an early warning tool for athletes and support preventative interventions \cite{Blanchard2019}.
\end{enumerate} These objectives collectively address the development of a holistic system: from the core vision algorithm (objective 1) to its practical application in sports analytics (objective 2) and safety monitoring (objective 3). Achieving them would represent a significant step towards making outdoor sports motion capture ubiquitous and actionable. \subsection{Scope and Contributions}
The scope of this thesis encompasses the design, implementation, and evaluation of a deep learning-based framework for outdoor human motion analysis, with a particular focus on pose estimation and injury risk assessment in sports. The work does not aim to cover every aspect of human motion capture (for instance, detailed 3D reconstruction from multiple cameras is beyond our scope), but instead concentrates on an end-to-end 2D pose analysis pipeline that is feasible in real-world field settings. Within this scope, the thesis makes several original contributions to the state of the art: \begin{itemize}
\item \textbf{A novel lightweight multi-person pose estimation architecture.} We propose a new deep neural network tailored for efficient human pose estimation in videos. The architecture introduces a streamlined network backbone and an optimized multi-person keypoint detection strategy (drawing on a bottom-up approach) to significantly reduce computational complexity. In our design, we prioritize model lightweightness and speed – for example, by using a smaller CNN feature extractor and efficient feature fusion – while maintaining accuracy comparable to heavier models. This network is a core technical contribution, demonstrating that high-throughput real-time pose estimation for multiple people is achievable on standard hardware without specialized accelerators. \item \textbf{Dataset adaptation and augmentation for outdoor sports.} To train and fine-tune the proposed model, we curate and adapt datasets that reflect the target domain of outdoor sports. This contribution includes the integration of existing human pose datasets (such as COCO, MPII, or AI Challenger) with new data and annotations specific to sports scenarios. We augment the training data with realistic variations (e.g., simulating different sunlight conditions, camera angles, and occlusions) to improve the model’s robustness to outdoor environments. Where existing data was insufficient, we have incorporated video frames from actual sports activities (e.g., soccer matches, sprinting drills) and annotated the athlete poses to create a more specialized benchmark for this project. This dataset effort ensures that the pose estimation network is explicitly targeted to outdoor sports context, which is a contribution distinct from prior work that mainly evaluates on general-purpose images. \item \textbf{Integrated sports performance analysis pipeline.} We develop a pipeline that takes the output of the pose estimation model (the time-series of detected joint positions) and computes key biomechanical metrics for analysis. This includes algorithms for deriving joint angles, detecting events (like foot strike or jump takeoff), and summarizing performance indicators (for instance, stride length, jump height, or limb symmetry). Moreover, we incorporate domain knowledge from sports science to interpret these metrics in a meaningful way. As a novel contribution, we demonstrate how a camera-based system can emulate some functions of a biomechanics lab, by providing coaches/athletes with measurable outcomes. The pipeline is validated through case studies, such as analyzing a runner’s gait or a team’s formations, illustrating the practical utility of our approach. \item \textbf{Injury risk assessment methodology using pose data.} As part of the analysis, the thesis contributes a methodology for using pose-derived features to predict injury risk. We identify several known risk factors (e.g., abnormal joint angles, asymmetries, or high-impact kinematics) and devise a scheme to automatically monitor these via the estimated poses. This may involve a simple threshold-based alert system for certain dangerous movements, or a learned classifier that has been trained to recognize patterns preceding injuries. We present an initial proof-of-concept injury risk model that operates on the pose time-series, and we evaluate its indications against expert assessments or literature-established risk criteria. This contribution is exploratory in nature, bridging computer vision with sports medicine by showing how pose estimation can facilitate preventative injury screening. \item \textbf{Empirical evaluation in real-world settings.} Lastly, we contribute a thorough evaluation of the entire system in both controlled tests and real-world outdoor scenarios. We report quantitative results for the pose estimation component, including accuracy on benchmark datasets and runtime performance (FPS) on various devices, to verify the “lightweight” claim. Additionally, we showcase the system operating on actual sports video footage in real time, demonstrating its deployment potential. This includes measuring the system’s performance when multiple athletes are in view and under different environmental conditions. The real-time inference capability and its implications are discussed, highlighting how our approach could be used on-the-fly during training sessions or competitions. By validating the approach outside of a lab, we solidify the contribution of a practically viable solution for outdoor human motion capture.
\end{itemize} In summary, the thesis delivers a complete framework that advances both the methodological aspect (with a new pose estimation model) and the applied aspect (with an analytics tool for sports performance and safety). The contributions lie in improving accessibility (camera-only, minimal equipment), focusing on an under-served context (outdoor, multi-person sports), and pushing towards real-time, actionable insights. Together, these contributions aim to illustrate a path forward for ubiquitous motion analysis that can benefit sports practitioners without the need for expensive infrastructure. \subsection{Thesis Structure}
The remainder of this thesis is organized as follows. Chapter 2: Literature Review surveys the relevant background in depth, covering existing motion capture technologies and related work in human pose estimation and sports biomechanics. We review prior art on traditional motion capture systems, wearable sensor approaches, and the evolution of vision-based pose estimation networks. This includes discussion of multi-person pose estimation algorithms and previous applications of these techniques in sport performance analysis and injury prediction, thereby positioning our research in the context of the state-of-the-art. Chapter 3: Methodology details the design of the proposed system. It describes the architecture of our lightweight multi-person pose estimation network, the data preparation and training procedure, and the development of the sports analysis and injury risk modules. We provide technical specifics and rationale for design choices, as well as any theoretical underpinnings for our approach. Chapter 4: Experiments and Results presents the empirical evaluation of our work. We first evaluate the pose estimation model on benchmark datasets and our curated sports dataset, reporting accuracy and speed metrics. Then we demonstrate the system’s performance on real-world outdoor sports videos, and analyze the results of the sports performance metrics and injury risk assessment in these scenarios. Quantitative results and qualitative examples are given to validate the effectiveness of the approach. Chapter 5: Discussion provides an in-depth discussion of the findings, implications, and limitations of the proposed approach. We interpret how well the objectives were met, discuss any constraints (such as conditions where the pose estimation struggles), and relate our results back to the literature. We also consider the practicality of deploying this system in live sports settings and any ethical or privacy considerations of video-based tracking. Finally, Chapter 6: Conclusion and Future Work concludes the thesis by summarizing the contributions and key results, and outlining potential directions for future research. This includes suggestions for improving the pose estimation accuracy (perhaps through integration of depth or multiple cameras for 3D estimation), expanding the injury prediction component with more data or advanced models, and broadening the system to other domains beyond sports. Through these chapters, the thesis builds a cohesive narrative from motivation and theory to implementation and application. Overall, we aim to demonstrate that lightweight multi-person pose estimation, coupled with domain-specific analysis, can significantly advance outdoor human motion analysis and injury risk prediction, making sophisticated motion capture techniques more accessible than ever before. \medskip\noindent\textit{Keywords:} Human Pose Estimation, Sports Biomechanics, Motion Capture, Multi-Person, Real-Time, Injury Risk Prediction, Deep Learning, Computer Vision.