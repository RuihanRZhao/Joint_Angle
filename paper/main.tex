\documentclass[a4paper,12pt]{article}
\usepackage[a4paper, margin=1in]{geometry}
\usepackage{fancyhdr}
\usepackage{graphicx}
\usepackage{ulem}
\usepackage{titlesec}
\usepackage{listings}
\usepackage{tocloft}
\usepackage{natbib} % For Harvard-style citations

% Font settings
\usepackage{helvet}
\renewcommand{\familydefault}{\sfdefault}

% Hyperref package for clickable links
\usepackage{hyperref}
\hypersetup{
    colorlinks=true,
    linkcolor=black,
    filecolor=black,      
    urlcolor=black,
    pdftitle={Your Project Title},
    pdfpagemode=FullScreen,
}

% Set up the fancy header
\fancypagestyle{main}{%
  \fancyhf{}
  \fancyhead[L]{\fontsize{9}{11}\selectfont\textbf{Project Title}}
  \fancyhead[R]{\fontsize{9}{11}\selectfont\textbf{Student Name}}
  \fancyfoot[C]{\fontsize{9}{11}\selectfont\thepage}
  \renewcommand{\headrulewidth}{0.4pt}
}

% Redefine plain style for consistent page numbering
\fancypagestyle{plain}{%
  \fancyhf{}
  \fancyfoot[C]{\fontsize{9}{11}\selectfont\thepage}
  \renewcommand{\headrulewidth}{0pt}
}

% Command to remove header and footer from specific pages
\newcommand{\removeheaderandfooter}{
  \pagestyle{empty}
  \fancyhead{}
  \fancyfoot{}
  \renewcommand{\headrulewidth}{0pt}
}

% Section formatting
\titleformat{\section}
  {\normalfont\huge}
  {\normalfont Chapter \thesection:}
  {0.25em}
  {\bfseries}

\titleformat{name=\section,numberless}
  {\normalfont\huge\bfseries}
  {}{0em}{}

% Code listing style
\lstset{
  basicstyle=\ttfamily\small,
  breaklines=true,
  frame=single
}

% Table of Contents formatting
\renewcommand{\cfttoctitlefont}{\huge\bfseries}
\renewcommand{\cftaftertoctitle}{\par\noindent\rule{\textwidth}{0.4pt}\vspace{3pt}}

% Remove bold from TOC entries
\renewcommand{\cftsecfont}{\normalfont}
\renewcommand{\cftsubsecfont}{\normalfont}
\renewcommand{\cftsubsubsecfont}{\normalfont}

% Modify section entries in TOC
\renewcommand{\cftsecpresnum}{Chapter }
\renewcommand{\cftsecaftersnum}{:}
\setlength{\cftsecnumwidth}{5em}

% Define and set page number font size in TOC to 9pt
\newcommand{\cftpagefont}{}
\renewcommand{\cftpagefont}{\fontsize{9}{11}\selectfont}

\begin{document}

% Start page numbering from 1, but don't display it
\pagenumbering{arabic}

% Title page
\begin{titlepage}
\removeheaderandfooter
\noindent
\begin{minipage}[t][0.95\textheight][t]{0.48\textwidth}
\raggedright
{\fontsize{15}{18}\selectfont School of Electronic Engineering and Computer Science}
\vfill
% Final Year text and University logo at the bottom
{\fontsize{15}{18}\selectfont 
Final Year\\
Undergraduate Project 2024/25
}
\vspace{4em}\\
\includegraphics[width=0.6\linewidth]{qmul_logo.png} % Adjust path to logo file
\end{minipage}
\hfill
\vrule depth 0.95\textheight width 0.4pt
\hfill
\begin{minipage}[t][0.95\textheight][t]{0.48\textwidth}
\raggedright
{\fontsize{14}{17}\selectfont
\textbf{Interim}\\ % delete as appropriate
\textbf{Programme of study:}\\
BSc. Computer Science\\[4em]
}
{\fontsize{20}{24}\selectfont \uline{\textbf{Project Title:}}\\
\textbf{Enhancing Accuracy of Human Skeletal Angle Measurement Using Stereo Vision and Multi-Camera Systems}\\[4em]
{\fontsize{14}{17}\selectfont
\textbf{Supervisor:}\\
Shanxin Yuan\\[4em]
\textbf{Student Name:}\\
Ruihan Zhao\\[4em]
\vfill
Date: 2025/May/12
}}
\end{minipage}
\end{titlepage}

% Set page counter to 2 to account for the title page
\setcounter{page}{2}

\clearpage
\removeheaderandfooter
\section*{Abstract}
In recent years, there has been a growing demand for lightweight human pose estimation models that can run efficiently on edge devices with limited compute resources. Various strategies have been explored to maintain accuracy while reducing model complexity, such as knowledge distillation and the design of efficient neural network architectures. In this undergraduate thesis project, we attempted to develop a highly compact 2D pose estimator for edge deployment by combining several state-of-the-art ideas: a Cross Stage Partial Network (CSPNet) backbone for efficient feature extraction, Squeeze-and-Excitation (SE) attention modules to recalibrate feature channels, and a SimCC-based keypoint regression head to replace conventional heatmaps.

Unfortunately, the resulting model failed to learn effective keypoint representations, achieving an extremely low Average Precision (AP$_{50}$ \< 1\%) on COCO-style person keypoint data. We identify several factors behind this negative outcome: lack of ImageNet pretraining for the small backbone, insufficient training data leading to underfitting, and high sensitivity of SimCC hyperparameters. These issues – compounded by the difficulty of optimizing a novel architecture without established best practices – resulted in the model failing to converge to reasonable accuracy.

Despite the disappointing results, this project offers valuable lessons. It underscores the importance of transfer learning and adequate data when training lightweight models for pose estimation, as well as careful validation of new techniques like SimCC. Crucially, it highlights that negative results have scientific value, informing future research toward more effective strategies for lightweight human pose estimation.


\clearpage
\pagestyle{plain}
\tableofcontents

\clearpage
\pagestyle{main}
% ———————————— 第一章:Introduction ————————————
\chapter{Introduction}
\section{Background}
Human pose estimation (HPE) aims to locate a set of anatomical keypoints (e.g., joints) on the human body from images or video. Early methods treated this as a graphical model problem, using hand-crafted features and part-based models. A seminal approach was the pictorial structures framework, which represents each limb as a deformable part connected in a tree and infers joint locations by optimizing pairwise spatial relationships \cite{Felzenszwalb2005}. While effective in controlled settings, these classical methods struggled with the high variability of real-world poses, occlusions, and complex backgrounds.

The introduction of deep learning dramatically advanced HPE accuracy. Toshev and Szegedy \cite{Toshev2014DeepPose} proposed DeepPose, the first end-to-end convolutional network to directly regress joint coordinates. However, direct coordinate regression proved challenging to train for high precision. Subsequent works reframed pose estimation as dense heatmap prediction for each keypoint, yielding substantial gains in localization accuracy \cite{Tompson2015}. Fully convolutional architectures such as the Stacked Hourglass network \cite{Newell2016} further improved performance by iteratively refining predictions over multiple stages. More recently, High-Resolution Networks (HRNet) maintained high-resolution feature maps throughout the network and achieved state-of-the-art accuracy by repeatedly fusing multi-scale information \cite{Sun_2019_CVPR}. These deep models now surpass traditional approaches by wide margins on benchmarks such as MPII and COCO \cite{Lan2023}.

Building on the success of convolutional networks, researchers have also explored transformer-based architectures for pose estimation. By leveraging self-attention mechanisms, transformer models can capture long-range dependencies between body joints. For example, TransPose and follow-on works integrate global context and have demonstrated competitive accuracy while simplifying the grouping of multi-person poses \cite{Stoffl2021}. Although transformers further close the accuracy gap, they often introduce additional computational overhead, rendering them less suitable for resource-constrained devices.

Extending to multi-person scenarios, HPE methods follow either a top-down or bottom-up paradigm. Top-down methods first detect each person with a general object detector (e.g., Mask R-CNN) and then apply a single-person pose estimator to each crop \cite{He2017}. This yields high per-person accuracy, but runtime grows linearly with the number of people. Bottom-up methods detect all keypoints for all people in a single pass and then assemble them into individual poses \cite{Cao_2017_CVPR}. Bottom-up pipelines have runtime largely independent of the number of people, making them attractive for crowded or real-time settings, though they often lag slightly behind top-down methods in raw accuracy \cite{Dubey2023PoseSurvey}. Hybrid approaches balance these trade-offs by learning grouping as part of the network.

Despite the remarkable accuracy of modern deep learning models, most state-of-the-art HPE networks are extremely large and computationally expensive. For instance, HRNet-W48 contains over 60 million parameters, and OpenPose’s two-branch network requires tens of GFLOPs per image. Such complexity hinders deployment on edge devices (e.g., mobile phones, embedded cameras) where memory, compute, and power budgets are strictly limited. Applications such as mobile fitness tracking, augmented reality, and human–robot interaction demand on-device pose estimation to ensure low latency, data privacy, and offline operation. Consequently, there is a pressing need for methods that deliver competitive pose estimation accuracy while operating within tight edge-device constraints.

This thesis investigates one such approach: combining an efficient convolutional backbone (CSPNet), compact attention modules (SE blocks), and a streamlined coordinate-based output representation (SimCC) to design a lightweight, edge-ready human pose estimator.


\section{Problem Statement}
The goal of this thesis is to design and evaluate a compact, efficient human pose estimation model suitable for real-time deployment on resource-constrained devices. In particular, we target a 2D pose estimator that can run on edge hardware (such as mobile or embedded platforms) with limited computational power, without offloading to powerful GPUs or cloud servers. The research question can be summarized as: How can we achieve high-accuracy human pose estimation with a lightweight model architecture that meets the speed and memory constraints of edge devices?

To address this problem, our approach integrates several techniques aimed at reducing network complexity while preserving accuracy. First, we adopt a Cross Stage Partial Network (CSPNet) architecture for the backbone feature extractor \citep{Wang2020CSPNet}. CSPNet is a design that partitions feature maps at each stage of the network into two parts – one part undergoes computation in the next layer while the other bypasses it – and then merges them. This effectively reduces duplicate gradient computations and allows a deep network to be thinner (fewer channels) without losing representational power. By incorporating CSPNet ideas into the pose model’s backbone, we aim to lower the number of parameters and FLOPs required for feature learning, which directly tackles the efficiency goal.

Second, we enhance the backbone with Squeeze-and-Excitation (SE) attention modules \citep{Hu2018SENet}. SE blocks perform a lightweight form of channel-wise attention by learning to reweight the feature channels according to their importance. This mechanism can improve the network’s feature quality and boost accuracy with only a modest increase in parameters. In a compact model, making the most of each channel is crucial; SE helps the network focus on informative features (for example, those corresponding to keypoint locations or human limb patterns) while suppressing less useful feature responses. We hypothesize that adding SE units will allow us to trim the model size further (since the remaining channels become more expressive) without sacrificing accuracy.

Third, our model employs a recently proposed keypoint representation called SimCC (Simple Coordinate Classification) \citep{Li2022SimCC}. Traditional pose estimators predict heatmaps on a high-resolution output grid (e.g. $64\times64$ or $128\times128$ per keypoint) and then find the maximum – this process is both memory- and compute-intensive due to the upsampling and large heatmap tensors. SimCC offers an alternative by reformulating pose estimation as two independent classification tasks for the $x$ and $y$ coordinates of each joint. In practice, the network produces probability distributions over possible $x$ positions and $y$ positions (within the image frame) instead of dense 2D heatmaps. This discretized coordinate classification achieves sub-pixel localization accuracy while avoiding the need for transpose convolutions or large output layers for heatmaps \citep{Li2022SimCC}. By using SimCC in our model’s keypoint head, we expect to significantly reduce the computation in the output stage and eliminate post-processing steps, thus streamlining the overall pipeline.

In summary, the project’s objective is to develop a pose estimation model that fuses these components – a CSPNet-based efficient backbone, SE attention modules, and a SimCC keypoint prediction head – to deliver high accuracy with low computational cost. We will measure success in terms of model size (parameter count), speed (inferences per second on a given device), and accuracy on standard pose benchmarks. The ideal outcome is a trained model that approaches the accuracy of state-of-the-art models on datasets like COCO, while being small and fast enough to run in real-time on a typical edge device. This would demonstrate a viable solution for real-world deployments of human pose estimation in scenarios where computing resources are limited.


\section{Thesis Structure}
The remainder of this thesis is organized as follows. Chapter 2 provides a review of relevant literature and technical background related to human pose estimation. We discuss previous work on pose estimation methods (both top-down and bottom-up approaches in more detail), as well as techniques for model efficiency and lightweight neural networks that inform our approach. Chapter 3 describes the design and methodology of the proposed lightweight pose estimation model. We detail the architecture of our system, including the integration of the CSPNet backbone, SE attention blocks, and the SimCC-based keypoint prediction strategy. This chapter also defines the model’s training strategy and any design choices or hyperparameters tuned during development. Chapter 4 covers the implementation and experimental setup, explaining how the model was implemented (e.g. software frameworks, training dataset preparation) and the evaluation protocol. We describe the hardware used for experiments, the datasets and metrics for evaluating pose estimation accuracy, and any comparative baselines or ablation studies conducted to assess each component’s impact. Chapter 5 then presents the results and evaluation of our approach. We report quantitative results such as accuracy (e.g. PCKh or AP on a keypoint benchmark) and runtime performance, and analyze these results in comparison to existing methods. We also discuss qualitative examples to illustrate the model’s strengths and failure cases. Finally, Chapter 6 concludes the thesis with a summary of our findings and the contributions of this project. We reflect on the extent to which the research objectives were met and discuss the limitations of our approach. We also outline possible future work and improvements, such as refining the model architecture further, exploring additional optimizations, or extending the approach to 3D pose estimation, which could be investigated to continue advancing lightweight human pose estimation for edge applications.

% ———————————— 第二章:Literature Review ————————————
\chapter{Literature Review}
\section{Human Pose Estimation}
% <经典方法与深度学习进展>

\section{Lightweight Network Architectures}
% <MobileNet, CSPNet, EfficientNet 等>

\section{Model Compression Techniques}
% <Pruning, Quantization, Knowledge Distillation 等>

\section{SimCC Method}
% <SimCC 原理与在姿态估计中的应用>

% ———————————— 第三章:Methodology ————————————
\chapter{Methodology}
\section{Network Architecture}
% <CSP backbone + SE modules 设计>

\section{SimCC Regression Head}
% <SimCC 如何集成、参数设置说明>

\section{Data Processing and Augmentation}
% <COCO 单人裁剪、增强方式等>

\section{Training Setup}
% <优化器、损失函数、超参、工具链 (PyTorch, W\&B)>

% ———————————— 第四章:Experimental Setup ————————————
\chapter{Experimental Setup}
\section{Dataset}
% <COCO 2017 及单人裁剪策略说明>

\section{Evaluation Metrics}
% <mAP, AP50, PCK 等指标定义>

\section{Tools and Environment}
% <硬件/软件环境、日志工具>

% ———————————— 第五章:Results and Failure Analysis ————————————
\chapter{Results and Failure Analysis}
\section{Quantitative Results}
% <训练/验证指标表格与曲线>

\section{Qualitative Visualizations}
% <关键点预测示例图 (红/绿骨架对比) >

\section{Failure Discussion}
% <分析主要失败原因:数据、预训练缺失、超参、SimCC 参数敏感等>

% ———————————— 第六章:Discussion ————————————
\chapter{Discussion}
\section{Comparison with Prior Work}
% <与文献中轻量或大型模型对比,差距原因>

\section{Lessons Learned}
% <从失败中得到的方法论与实践教训>

% ———————————— 第七章:Conclusion and Future Work ————————————
\chapter{Conclusion and Future Work}
\section{Summary}
% <项目总结与主要收获>

\section{Key Takeaways}
% <核心结论点>

\section{Future Directions}
% <下一步改进方案:预训练、更多数据、蒸馏、分阶段训练等>

% ———————————— 参考文献 ————————————
\clearpage
\bibliographystyle{apalike}  % 或 qmul 推荐样式
\bibliography{references}    % 将引用条目写入 references.bib

% ———————————— 附录 ————————————
\appendix
\chapter{Code Structure}
% <项目目录与关键脚本说明>

\chapter{W\&B Training Logs}
% <可附训练曲线截图或日志摘要>

\chapter{Configuration Files}
% <超参数与配置详情>

\end{document}