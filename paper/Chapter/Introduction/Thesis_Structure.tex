The remainder of this thesis is organized as follows. Chapter 2 provides a review of relevant literature and technical background related to human pose estimation. We discuss previous work on pose estimation methods (both top-down and bottom-up approaches in more detail), as well as techniques for model efficiency and lightweight neural networks that inform our approach. Chapter 3 describes the design and methodology of the proposed lightweight pose estimation model. We detail the architecture of our system, including the integration of the CSPNet backbone, SE attention blocks, and the SimCC-based keypoint prediction strategy. This chapter also defines the model’s training strategy and any design choices or hyperparameters tuned during development. Chapter 4 covers the implementation and experimental setup, explaining how the model was implemented (e.g. software frameworks, training dataset preparation) and the evaluation protocol. We describe the hardware used for experiments, the datasets and metrics for evaluating pose estimation accuracy, and any comparative baselines or ablation studies conducted to assess each component’s impact. Chapter 5 then presents the results and evaluation of our approach. We report quantitative results such as accuracy (e.g. PCKh or AP on a keypoint benchmark) and runtime performance, and analyze these results in comparison to existing methods. We also discuss qualitative examples to illustrate the model’s strengths and failure cases. Finally, Chapter 6 concludes the thesis with a summary of our findings and the contributions of this project. We reflect on the extent to which the research objectives were met and discuss the limitations of our approach. We also outline possible future work and improvements, such as refining the model architecture further, exploring additional optimizations, or extending the approach to 3D pose estimation, which could be investigated to continue advancing lightweight human pose estimation for edge applications.