\begin{itemize}
  \item \textbf{Transfer learning and data sufficiency.} Leveraging pretrained backbones on large‐scale image datasets such as ImageNet \cite{Mathis2021Pretraining} or domain‐specific corpora ensures that low‐level filters and mid‐level representations are already optimized before fine‐tuning. Coupling this with an adequately sized, diverse training set prevents the network from memorizing narrow pose variations and underpins its ability to generalize across different subjects, viewpoints, and lighting conditions. Studies have shown that doubling the dataset size can improve out‐of‐domain mAP by over 8\% \cite{Andriluka2014MPII}.
  \item \textbf{Data augmentation and training schedule robustness.} Beyond basic geometric transforms, advanced augmentation techniques—such as AutoAugment \cite{Cubuk2019AutoAugment}, CutMix \cite{Yun2019CutMix}, and MixUp \cite{Zhang2018MixUp}—introduce synthetic diversity that bridges domain gaps. A carefully designed learning rate schedule, for example cosine annealing with warm restarts \cite{Loshchilov2016SGDR}, stabilizes convergence and prevents overfitting plateaus. Empirical results indicate that such schedules can accelerate training by 20\% and reduce validation loss variance by 15\%.
  \item \textbf{Integration of novel output representations.} Techniques like SimCC \cite{Li2022SimCC} and heatmap regression have distinct trade‐offs in computational cost and localization precision. SimCC’s classification‐based head eliminates large 2D heatmaps, saving over 50\% of FLOPs, but requires precise tuning of bin counts and smoothing parameters. Hybrid approaches—initializing with coarse heatmap supervision and subsequently refining with SimCC—have demonstrated improved performance, yielding a 2–4\% boost in AP@50 on COCO benchmarks.
  \item \textbf{Staged training pipelines and student–teacher frameworks.} Adopting a curriculum learning strategy, where the model is first trained with mean‐squared‐error on heatmaps before transitioning to KL‐divergence on SimCC distributions, stabilizes gradient flow and speeds up convergence \cite{Bengio2009Curriculum}. Integrating knowledge distillation—trusted teacher models supplying soft targets—can further enhance performance, with reported gains of up to 6\% mAP over vanilla training \cite{Li2021OKD,Hinton2015Distill}.
\end{itemize}