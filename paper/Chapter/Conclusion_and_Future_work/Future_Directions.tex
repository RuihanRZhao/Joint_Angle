To address the shortcomings identified, the following roadmap is proposed:

\begin{enumerate}
  \item \textbf{Use an ImageNet-pretrained backbone}, such as MobileNetV3 or EfficientNet, to leverage learned visual features \cite{Howard2017}. Progressive unfreezing techniques should be employed: first train only the newly added task-specific layers for a few epochs, then gradually unfreeze intermediate convolutional blocks to fine-tune mid-level representations. This staged fine-tuning preserves high-quality low-level filters, accelerates convergence, and mitigates catastrophic forgetting in transfer learning.
  \item \textbf{Expand and augment the dataset}, integrating full COCO keypoint annotations, external pose datasets like MPII \cite{Andriluka2014MPII} and CrowdPose \cite{Li2019CrowdPose}, and synthetic samples generated via generative adversarial networks. A comprehensive augmentation pipeline should include random rotations (±45°), scaling jitter (0.75×–1.25×), horizontal flips, copy-paste occlusion using CutMix \cite{Yun2019CutMix}, and photometric distortions following AutoAugment \cite{Cubuk2019AutoAugment}. These measures diversify training samples and improve robustness to real-world variations.
  \item \textbf{Optimize SimCC hyperparameters}, performing systematic search (grid or Bayesian) over bin counts (e.g., 4, 6, 8) and Gaussian standard deviations (1.0, 1.5, 2.0). Automated tuning frameworks like Optuna can efficiently explore this space within limited compute budgets. Additionally, a resolution curriculum can progressively increase classification granularity by starting with coarse bins and refining toward finer discretizations over epochs.
  \item \textbf{Incorporate knowledge distillation}, using a high-capacity teacher model to guide the small student network. Both offline distillation (transferring soft distributions) and online distillation (joint training with mutual objectives) should be evaluated \cite{Hinton2015Distill,Li2021OKD}. Distillation typically yields 3–6% mAP improvement for the student without additional inference cost.
  \item \textbf{Adopt progressive training}, utilizing curriculum learning \cite{Bengio2009Curriculum} where simple poses (e.g., frontal views) are learned first, followed by complex or occluded ones. Dynamic resizing schedules, such as SGDR warm restarts \cite{Loshchilov2016SGDR}, can help the optimizer escape local minima and sustain high validation performance across cycles.
\end{enumerate}

Implementing these strategies is expected to yield a substantial performance improvement, targeting mid-60s AP on COCO while keeping the model size below 5M parameters.
