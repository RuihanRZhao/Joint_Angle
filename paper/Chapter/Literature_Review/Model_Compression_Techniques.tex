While selecting efficient architectures forms the foundation, the real magic happens when we apply compression techniques to streamline human pose estimation networks. Let's explore three powerful methods that help shrink models while maintaining their predictive prowess:

\begin{itemize}
\item \textbf{The Art of Pruning:} Imagine trimming a bonsai tree - skilled pruning removes redundant branches while preserving its essential shape. Similarly, neural network pruning \citep{Han2016} carefully eliminates less important weights or entire filters. The key lies in identifying parameter redundancy, creating leaner models through structured pruning (removing complete channels/layers). For pose estimation, this translates to excising neurons that contribute little to keypoint predictions. The payoff? Faster inference speeds without significant accuracy drops, much like our bonsai maintains its beauty despite being more compact.

\item \textbf{Precision Diet with Quantization:} Why carry heavy weights when lighter ones work just as well? Quantization \citep{Jacob2018} converts 32-bit floating-point numbers into 8-bit integers (or lower), dramatically slashing memory needs. This isn't just about storage - specialized hardware like NPUs thrive on these lightweight values, accelerating computations through efficient integer math. When applied to pose models, careful quantization (using calibration or quantization-aware training) maintains prediction quality while enabling deployment on resource-constrained devices. It's like converting a encyclopedia into a well-illustrated pocket guide - same knowledge, portable format.

\item \textbf{Knowledge Alchemy through Distillation:} In medieval apprenticeship traditions, masters transferred skills to novices through guided practice. Knowledge distillation \citep{Hinton2015Distill} works similarly, where a compact "student" model learns from a bulky but accurate "teacher". For pose estimation, this means training lightweight models to mimic the heatmap outputs of their sophisticated counterparts. The result? Small models that punch above their weight class, recovering accuracy lost during simplification. Think of it as capturing a master painter's technique in a quick sketch - less detail, same essence.
\end{itemize}

The true power emerges when combining these techniques like ingredients in a recipe. A common workflow might involve:

1. Training a robust pose network

2. Pruning its excess parameters

3. Converting to lightweight numerical formats

4. Final polishing through distillation

This layered approach creates models that balance speed and accuracy like a seasoned tightrope walker - achieving what seemed impossible for edge deployment. Through strategic compression, we transform computational heavyweights into nimble performers ready for real-time action on devices from smartphones to embedded systems.