An important recent development in pose estimation is the SimCC method, which reformulates keypoint prediction as a coordinate classification task rather than a heatmap regression task. SimCC (Simple Coordinate Classification) treats the $x$ and $y$ coordinates of each keypoint as discrete classes to be predicted \citep{Li2022SimCC}. In practice, the image is divided into a grid, and two classifier heads (one for horizontal position and one for vertical position) output probability distributions over quantized coordinate bins. This encoding enables the network to achieve sub-pixel localization accuracy by subdividing pixels into finer bins, effectively addressing the quantization error inherent in standard heatmap approaches. Unlike heatmaps, which produce a blobby activation that must be post-processed to obtain coordinates, SimCC yields coordinate estimates by simply taking the argmax of the classification outputs along each axis. As a result, it removes the need for complex post-processing like Gaussian fitting or offset regression to refine keypoint locations.

The SimCC formulation offers several advantages over conventional heatmap-based pose estimation. First, by eliminating high-resolution heatmap predictions, SimCC can forego the deep upsampling layers that many state-of-the-art networks use to generate fine-grained heatmaps \citep{Li2022SimCC}. This substantially reduces the computation: for example, replacing a heatmap head with SimCC was reported to cut more than 50\% of the FLOPs in a ResNet-50 based pose model \citep{Li2022SimCC}. Second, SimCC’s direct coordinate output avoids the quantization drift caused by low-resolution feature maps — a known limitation of heatmap methods when input resolution is limited. Indeed, SimCC has demonstrated superior accuracy to heatmap baselines, particularly in low-resolution settings where heatmap methods typically struggle \citep{Li2022SimCC}. Third, the simplicity of SimCC’s two-classifier design makes it easy to integrate with different backbones (CNN or transformer) in an end-to-end framework. Experiments on standard benchmarks (COCO, MPII, CrowdPose) have shown that models using SimCC can match or exceed the accuracy of equivalent models using heatmaps, all while yielding a simpler and more efficient pipeline \citep{Li2022SimCC}. This method exemplifies the ongoing innovation in pose estimation techniques aimed at improving efficiency without sacrificing precision.