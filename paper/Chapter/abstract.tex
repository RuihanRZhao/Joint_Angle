The quest for portable pose estimation systems has intensified as edge computing reshapes AI deployment landscapes. Engineers worldwide now face a paradoxical challenge: how to shrink models for resource-constrained devices without sacrificing their ability to "see" human movement accurately. While knowledge distillation and neural architecture innovations have pushed boundaries, our undergraduate team discovered firsthand how fragile this balancing act can be.

This thesis project aimed to craft a minimalist 2D pose detector using three cutting-edge components: a CSPNet backbone for lean feature extraction, SE attention modules acting as "neural tuning forks" to amplify useful signals, and SimCC regression heads promising computational savings over traditional heatmaps. On paper, this architectural cocktail seemed potent. In practice, our model developed what we grimly dubbed "robotic arthritis" – joints detection accuracy plummeted to a dismal (AP$_{50}$~$<1\%$) on COCO-style benchmarks.

Post-mortem analysis revealed multiple missteps. First, skipping ImageNet pretraining for our trimmed-down backbone proved catastrophic, like sending a novice chef to cook Michelin meals with only five ingredients. Second, our training dataset resembled a sparse sketchbook rather than the rich visual encyclopedia these systems crave. Most crucially, the SimCC module – while theoretically elegant – demanded hyperparameter precision rivaling Swiss watchmaking, with catastrophic sensitivity to minor miscalibrations.

Yet within this technical shipwreck glimmers scientific treasure. The experience reinforces three hard truths: 1) Transfer learning isn’t optional armor but essential survival gear for lightweight models, 2) Data starvation induces algorithmic malnutrition, and 3) Novel architectures demand "training wheels" – phased implementation rather than all-or-nothing adoption. Ironically, our failed convergence patterns now serve as cautionary waypoints for peers navigating similar terrain.

This painful yet enlightening journey underscores a growing realization in edge AI research: Negative results aren’t dead ends but rather signposts written in inverse. They force us to question assumptions about what truly makes compact models tick – lessons no textbook can adequately convey.