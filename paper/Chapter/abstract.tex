In recent years, there has been a growing demand for lightweight human pose estimation models that can run efficiently on edge devices with limited compute resources. Various strategies have been explored to maintain accuracy while reducing model complexity, such as knowledge distillation and the design of efficient neural network architectures. In this undergraduate thesis project, we attempted to develop a highly compact 2D pose estimator for edge deployment by combining several state-of-the-art ideas: a Cross Stage Partial Network (CSPNet) backbone for efficient feature extraction, Squeeze-and-Excitation (SE) attention modules to recalibrate feature channels, and a SimCC-based keypoint regression head to replace conventional heatmaps.

Unfortunately, the resulting model failed to learn effective keypoint representations, achieving an extremely low Average Precision (AP$_{50}$ \< 1\%) on COCO-style person keypoint data. We identify several factors behind this negative outcome: lack of ImageNet pretraining for the small backbone, insufficient training data leading to underfitting, and high sensitivity of SimCC hyperparameters. These issues – compounded by the difficulty of optimizing a novel architecture without established best practices – resulted in the model failing to converge to reasonable accuracy.

Despite the disappointing results, this project offers valuable lessons. It underscores the importance of transfer learning and adequate data when training lightweight models for pose estimation, as well as careful validation of new techniques like SimCC. Crucially, it highlights that negative results have scientific value, informing future research toward more effective strategies for lightweight human pose estimation.
