Several practical and methodological lessons emerged from this work:

\begin{itemize}
  \item \textbf{Pretraining is essential.} Initializing the CSP backbone with ImageNet‐pretrained weights dramatically accelerated convergence and improved early feature learning by seeding low‐level edge, color, and texture detectors that random initialization cannot match \cite{Mathis2021Pretraining}. Without such transfer, our model exhibited unstable gradient flows in early epochs and required significantly more iterations to reduce loss. Moreover, domain‐specific pretraining—such as on human‐segmentation or action‐recognition datasets—provides mid‐level feature adaptation that can enhance pose estimation accuracy by up to 15\% in out‐of‐domain tests \cite{Radenovic2018Revisiting}. For future iterations, we recommend progressive layer freezing, wherein pretrained layers are locked during initial training phases and gradually unfrozen for fine‐tuning once high‐level heads stabilize.

  \item \textbf{Data scale must match model capacity.} While 134 214 single‐person crops provided a baseline, the network’s capacity far exceeded the diversity of poses, scales, and backgrounds in our dataset, leading to severe overfitting \cite{Dubey2023PoseSurvey}. Incorporating additional datasets—such as MPII \cite{Andriluka2014MPII} and CrowdPose \cite{Li2019CrowdPose}—would supply rare or extreme postures and crowd scenarios. Aggressive augmentation pipelines that include random rotation, scaling jitter, photometric distortion, and copy‐paste occlusion help simulate real‐world variability and mitigate overfitting \cite{Shorten2019Augmentation}. Automated policies like AutoAugment or RandAugment can further optimize augmentation mixes without extensive manual tuning.

  \item \textbf{Innovations require careful tuning.} Novel modules like the SimCC head introduce hyperparameters—bin count, Gaussian label width, and upsampling stride—that directly influence localization fidelity \cite{Li2022SimCC}. Our ablation studies demonstrated that raising the bin count from 4 to 6 reduced discretization error by approximately 12\%, while adjusting the smoothing parameter $\sigma$ from 1.5 to 1.0 sharpened probability peaks and improved AP@50 by over 1\%. To systematically explore this space, automated tuning frameworks such as Bayesian optimization or Hyperband should be employed to balance accuracy gains against computational cost.

  \item \textbf{Training strategy matters.} A naive one‐stage, end‐to‐end training process may underutilize a model’s full capacity, particularly for lightweight architectures. Curriculum learning—starting with coarse heatmap objectives before introducing fine‐grained coordinate classification—stabilizes gradients in early stages and boosts final performance \cite{Bengio2009Curriculum}. Multi‐stage refinement pipelines, wherein a coarse heatmap model bootstraps the SimCC head, combine the robustness of dense supervision with the efficiency of coordinate classification \cite{Zhang2020LPN}. Additionally, knowledge distillation from a high‐capacity teacher network transfers structural priors and accelerates convergence, as demonstrated by both offline and online distillation strategies \cite{Hinton2015Distill,Li2021OKD}.
\end{itemize}

These lessons will guide the next iteration of our lightweight pose estimation pipeline, emphasizing robust initialization, scalable data engineering, principled hyperparameter exploration, and staged training curricula.
